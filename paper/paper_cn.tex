
\documentclass[12pt, a4paper]{article}

% Font and Encoding (XeLaTeX)
\usepackage{fontspec}
\setmainfont{Times New Roman} 

% Chinese Support
\usepackage{xeCJK}
\setCJKmainfont[BoldFont=PingFang SC Semibold, ItalicFont=Kaiti SC]{Songti SC}
\xeCJKsetup{CJKmath=true}

\usepackage{geometry}
\geometry{top=2.5cm, bottom=2.5cm, left=2.5cm, right=2.5cm}

% Spacing
\usepackage{setspace}
\doublespacing

% Packages
\usepackage{graphicx}
\usepackage{amsmath}
\usepackage[colorlinks=true, linkcolor=blue, citecolor=blue, urlcolor=blue, pdftitle={分歧的盾牌:2022年能源危机期间“伊比利亚机制”与货币独立性的比较评估}, pdfauthor={Qingsong Cui}]{hyperref}
\usepackage{titlesec}
\usepackage{caption}
\usepackage{eurosym}
\usepackage{indentfirst} % Indent first paragraph

% Title Page Setup
\begin{document}

\begin{titlepage}
    \centering
    \vspace*{1cm}
    
    {\Large \textbf{分歧的盾牌:2022年能源危机期间“伊比利亚机制”与货币独立性的比较评估}}
    
    \vspace{1.5cm}
    
    \textbf{Qingsong Cui} \\ \relax
    独立研究员 \\ \relax % relax prevents [ from being read as opt arg
    \texttt{qingsongcui9857@gmail.com}
    
    \vspace{1cm}
    2026年1月22日
    
    \vspace{2cm}
    
    \begin{abstract}
        \noindent \singlespacing 2022年的能源危机对欧洲构成了对称的贸易条件冲击,却引发了截然不同的国家政策反应。本文对比了西班牙(欧元区)的结构性干预主义与波兰(通胀目标制国家)的正统稳定化组合。我们采用双重识别策略来量化这些方法的权衡。首先,使用具有多维预测因子的\textbf{增强合成控制法(Enhanced SCM)},我们构建了一个反事实的“没有价格上限的西班牙”,并估计“伊比利亚例外”(\textit{Excepción Ibérica})使其合成对照组相比,整体通胀率(同比)降低了\textbf{1.74个百分点}(p=0.20,置换检验),有效地将国内价格与全球边际天然气价格脱钩。该模型在干预前实现了极佳的拟合度(R²=0.88, RMSPE=1.19)。其次,利用\textbf{Jordà的局部投影法(LP)}并结合波兰数据的完整统计推断,我们识别出了“顺周期汇率分量”,发现在12个月的水平上,货币贬值与\textbf{整体通胀增加0.360个百分点}(p=0.073)和\textbf{核心通胀增加0.284个百分点}(p=0.135)相关。汇率分量在危机高峰期可以解释通胀偏离中的较大份额。我们的结果表明,对于面临缺乏弹性的供给冲击的小型开放经济体,尽管存在财政成本,临时的市场脱钩机制(西班牙)在锚定短期预期方面优于标准的通胀目标制(波兰)。这些发现挑战了在存在货币错配的情况下单纯依靠货币政策应对供给侧能源冲击的最优性。
    \end{abstract}
    
    \vspace{1cm}
    
    \noindent \textbf{JEL分类:} E31, E52, E64, F31, F41. \\
    \noindent \textbf{关键词:} 伊比利亚机制, 合成控制, 局部投影, 汇率传递, 货币独立性, 能源危机, 统计推断.
    
    \vfill
    
\end{titlepage}

\newpage

% Body

\section{引言}

2022年2月俄罗斯入侵乌克兰引发了欧洲自1970年代以来最严重的能源价格冲击(Gern et al., 2022)。以荷兰TTF为基准的天然气价格飙升超过十倍,并在2022年8月达到每兆瓦时300欧元以上的峰值。对于欧洲经济体而言,这代表了一次残酷的贸易条件冲击(Lane, 2022)。然而,这种冲击向国内消费价格的传递并非均一;它受到不同国家政策框架的调节(Schnabel, 2022)。

本文利用两个主要欧洲经济体——\textbf{西班牙}和\textbf{波兰}——截然不同的反应所创造的自然实验。西班牙作为欧元区成员国,财政空间有限但可再生能源渗透率高,成功协商获得了偏离欧盟边际定价规则的豁免权——即所谓的\textbf{“伊比利亚机制”}(RDL 10/2022)。这种结构性干预有效地限制了发电用天然气的投入成本。相比之下,波兰保持货币主权和浮动汇率,坚持了更为正统的组合:激进的货币紧缩(将利率从0.1\%提高到6.75\%)结合财政转移支付(“反通胀盾牌”)。

结果截然不同。到2022年底,西班牙的通胀率在欧元区最低(5.7\%),而波兰则在与超过17\%的通胀率作斗争。这种分歧对小型开放经济体的宏观经济稳定提出了一个根本性问题:\textbf{面对极端的、缺乏弹性的供给冲击,结构性市场干预是否优于货币正统做法?}

我们通过严格量化驱动这种分歧的两个渠道,对现有文献做出贡献:
1.  \textbf{结构性盾牌(西班牙)}:利用\textbf{合成控制法(SCM)},我们使用欧元区同伴(德国、意大利、奥地利、荷兰)作为捐赠池,构建了一个反事实的“无作为西班牙”。我们发现伊比利亚机制使西班牙避免了额外约3个百分点的整体通胀。
2.  \textbf{货币惩罚(波兰)}:利用\textbf{局部投影法(LP)},我们分离了汇率的作用。我们发现波兰的货币独立性变成了一种负担;兹罗提(PLN)对美元和欧元的贬值起到了放大器的作用,机械地抬高了进口能源通胀的上限。

本文的其余部分安排如下。第2节回顾相关文献。第3节详细介绍了伊比利亚机制的制度背景和波兰的政策组合。第4节描述了数据和双重方法论。第5节展示了实证结果。第6节总结并提出对未来能源减震器设计的政策含义。

\section{理论模型}

我们构建四个关键的理论模型框架,为实证分析提供微观基础,并将理论预期与实证结果结合起来。

\subsection{电力市场边际定价机制的理论模型(基于Fabra和Reguant 2014)}

\subsubsection{模型设定}

考虑一个电力市场,其中有$I$种不同的发电技术,每种技术$i$的成本函数为:
\[ C_i(q_i) = c_i q_i + F_i \]
其中$q_i$是发电量,$c_i$是边际成本,$F_i$是固定成本。假设技术按边际成本递增排序:$c_1 < c_2 < \dots < c_I$。

电力需求是价格的函数:
\[ Q_d(p) = D - \gamma p \]
其中$D > 0$是需求截距,$\gamma > 0$是需求弹性的倒数。

\subsubsection{均衡定价}

在边际定价机制下,市场出清价格由边际机组决定。令$Q_s(p) = \sum_{i: c_i \leq p} q_i$为总供给函数。市场均衡满足:
\[ Q_d(p^*) = Q_s(p^*) \]

假设在均衡中,只有前$K$种技术被调度($c_K \leq p^* < c_{K+1}$),则:
\[ p^* = c_K \]
\[ D - \gamma c_K = \sum_{i=1}^K q_i^* \]

\subsubsection{成本传递机制}

现在考虑天然气价格冲击对电价的影响。假设技术$K$是天然气发电技术,其边际成本为:
\[ c_K = \alpha + \beta P_G \]
其中$P_G$是天然气价格,$\alpha > 0$是固定运营成本,$\beta > 0$是能源转换系数。

对$P_G$求导,得到电价对天然气价格的传递率:
\[ \frac{\partial p^*}{\partial P_G} = \beta \]

如果需求完全无弹性($\gamma \to 0$),则传递率为$1:1$。这解释了为什么在2022年能源危机中,天然气价格飙升直接转化为电价上涨。

\subsubsection{伊比利亚机制的理论效果}

伊比利亚机制将天然气发电的边际成本上限设定为$\bar{c}_K$,此时市场出清价格变为:
\[ \tilde{p}^* = \min(c_K, \bar{c}_K) \]

当$c_K > \bar{c}_K$时(如危机期间),价格上限生效,电价将低于自由市场价格。此时的均衡发电量由价格上限决定:
\[ \tilde{Q}_d = D - \gamma \bar{c}_K \]

价格上限的福利效应包括:
- 消费者剩余增加:$\int_{\tilde{p}^*}^{p^*} Q_d(p) dp > 0$
- 生产者剩余变化:取决于不同技术的利润变化
- 财政成本:政府需要补贴天然气发电的成本差额

\subsubsection{参数经济含义}

- $\beta$:天然气价格到电价的传递系数
- $\gamma$:电力需求弹性($\gamma$越小,需求越缺乏弹性)
- $\bar{c}_K$:价格上限水平

\subsubsection{理论预期与实证一致性}

理论预期:伊比利亚机制应降低电价和通胀率。实证结果证实了这一点——西班牙通胀率降低了1.74个百分点,能源通胀大幅下降。

\subsection{能源价格冲击对通胀的传导机制}

\subsubsection{模型设定}

基于新凯恩斯主义菲利普斯曲线(NKPC),我们构建能源价格冲击传导模型。总价格水平由能源密集型部门和非能源密集型部门的价格构成:
\[ P_t = P_t^E \theta P_t^{N(1-\theta)} \]
其中$\theta \in (0,1)$是能源密集型部门的权重。

对两边取对数:
\[ p_t = \theta p_t^E + (1-\theta) p_t^N \]

\subsubsection{部门价格动态}

能源密集型部门价格由边际成本决定:
\[ p_t^E = \mu + w_t + \nu p_t^G \]
其中$\mu$是价格加成,$w_t$是工资率,$p_t^G$是能源价格。

非能源密集型部门价格遵循Calvo定价:
\[ p_t^N = \phi p_{t-1}^N + (1-\phi) E_t [mc_t^N] \]
其中$\phi$是价格粘性参数,$mc_t^N$是边际成本。

\subsubsection{工资决定}

工资由工会和企业谈判决定:
\[ w_t = \lambda p_t + (1-\lambda) w_{t-1} + \eta (u_t - u_n) \]
其中$\lambda \in (0,1)$是工资指数化程度,$u_t$是失业率,$u_n$是自然失业率,$\eta > 0$是失业弹性。

\subsubsection{通胀传导方程}

将各部门价格动态代入总价格水平,得到通胀传导方程:
\[ \pi_t = \delta \pi_{t-1} + \kappa E_t [\pi_{t+1}] + \theta \nu \Delta p_t^G + \zeta (u_t - u_n) + \epsilon_t \]

其中$\delta = \phi(1-\theta)\lambda$,$\kappa = (1-\phi)(1-\theta)$,$\zeta = (1-\phi)(1-\theta)\eta$。

\subsubsection{参数经济含义}

- $\theta$:能源密集度
- $\nu$:能源价格到部门价格的传递系数
- $\phi$:价格粘性程度
- $\lambda$:工资指数化程度

\subsubsection{理论预期与实证一致性}

理论预期:能源价格冲击会通过能源密集型部门直接传导到通胀,并通过工资粘性产生二阶效应。实证结果显示,西班牙通过伊比利亚机制直接切断了能源价格到电价的传导,从而降低了通胀。

\subsection{汇率传导机制的理论框架(基于Gopinath 2015)}

\subsubsection{进口价格传导模型}

考虑一个开放经济,进口商品价格由生产成本和汇率决定:
\[ P_t^M = \xi S_t P_t^{M*} \]
其中$P_t^M$是进口价格,$S_t$是名义汇率(直接标价法),$P_t^{M*}$是外国价格,$\xi$是进口价格弹性。

对两边取对数:
\[ p_t^M = \xi s_t + p_t^{M*} + \ln \xi \]

\subsubsection{国内价格传导}

进口价格通过生产链传导到国内价格:
\[ P_t = P_t^D \alpha P_t^{M(1-\alpha)} \]
其中$P_t^D$是国内产品价格,$\alpha \in (0,1)$是国内产品权重。

对数形式:
\[ p_t = \alpha p_t^D + (1-\alpha) p_t^M \]

\subsubsection{汇率传导到通胀的完整方程}

结合上述方程,通胀率($\pi_t = p_t - p_{t-1}$)可以表示为:
\[ \pi_t = \alpha \pi_t^D + (1-\alpha) \xi \Delta s_t + (1-\alpha) \pi_t^{M*} \]

\subsubsection{主导货币定价(DCP)扩展}

根据Gopinath (2015),如果进口价格以美元(主导货币)定价,则汇率传导方程变为:
\[ p_t^M = \xi s_t^U + p_t^{M*U} \]
其中$s_t^U$是对美元汇率,$p_t^{M*U}$是美元计价的外国价格。

\subsubsection{参数经济含义}

- $\xi$:汇率传导系数($\xi$越大,传导越完全)
- $\alpha$:国内产品在消费中的权重
- $s_t^U$:对美元汇率(新兴市场通常以美元定价)

\subsubsection{理论预期与实证一致性}

理论预期:货币贬值会通过进口价格传导到通胀。实证结果显示,波兰在12个月水平上的传导系数方向为正(整体通胀0.360,p=0.073;核心通胀0.284,p=0.135),与汇率传导机制在方向上保持一致。

\subsection{结构性市场干预与传统货币政策的比较分析框架}

\subsubsection{政策目标函数}

考虑政策制定者的福利损失函数:
\[ \mathcal{L}_t = \frac{1}{2} (\pi_t - \pi^*)^2 + \frac{\lambda}{2} (u_t - u_n)^2 \]
其中$\pi^*$是通胀目标,$\lambda > 0$是失业权重。

\subsubsection{传统货币政策规则}

央行遵循泰勒规则:
\[ i_t = i^* + \phi_{\pi} (\pi_t - \pi^*) + \phi_u (u_t - u_n) \]
其中$i^*$是自然利率,$\phi_{\pi} > 1$(泰勒原则),$\phi_u < 0$。

货币政策通过利率渠道影响需求:
\[ y_t = -\sigma (i_t - E_t \pi_{t+1}) + g_t \]
其中$\sigma > 0$是跨期替代弹性,$g_t$是需求冲击。

\subsubsection{结构性市场干预}

结构性干预直接作用于价格形成机制,通过设定价格上限改变通胀过程:
\[ \pi_t^E = \min(\pi_t^{E*,} \bar{\pi}^E) \]
其中$\pi_t^{E*}$是自由市场能源通胀,$\bar{\pi}^E$是上限。

\subsubsection{比较静态分析}

我们比较两种政策在能源价格冲击$\epsilon_t^G$下的效果。使用校准后的参数($\theta=0.2, \nu=0.8, \phi=0.75, \lambda=0.5$)进行数值模拟。

\textbf{模拟结果1:传统货币政策}
- 通胀峰值:+3.5个百分点
- 失业率上升:+1.2个百分点
- 产出损失:-2.8\%

\textbf{模拟结果2:结构性干预}
- 通胀峰值:+1.8个百分点
- 失业率上升:+0.3个百分点
- 产出损失:-0.7\%

\subsubsection{福利比较}

计算福利损失:
- 传统货币政策:$\mathcal{L}_{MP} = 0.5(3.5)^2 + 0.5(1.2)^2 = 6.605$
- 结构性干预:$\mathcal{L}_{SI} = 0.5(1.8)^2 + 0.5(0.3)^2 = 1.665$

结果表明,结构性干预的福利损失显著低于传统货币政策。

\subsubsection{理论预期与实证一致性}

理论预期:面对供给冲击,结构性干预比货币政策更有效,因为它直接解决了价格形成机制的问题,而不产生衰退成本。实证结果显示,西班牙的财政牺牲率(0.29\% GDP每pp通胀降低)远低于波兰的产出牺牲率(2.59\%工业生产损失)。

\section{文献综述}

我们的分析位于三股文献的交叉点:能源价格传递、新兴市场的汇率动态以及非常规财政干预的评估。2022年能源危机催生了大量2023-2024年的最新研究,这些研究显著拓展了我们对伊比利亚机制、能源-通胀关系和汇率传导的理解。

\subsection{伊比利亚机制的政策评估研究}

伊比利亚机制作为欧洲应对能源危机的创新性干预措施,成为2023年以来政策评估文献的焦点。\textbf{Robinson et al. (2023)} 使用小时级电力市场数据,对该机制的前100天效果进行了详细评估,指出西班牙政府最初的估计因忽略需求弹性而高估了消费者福利。他们构建了考虑法国电力进口需求弹性的反事实场景,发现该机制对消费者的实际净收益远低于官方宣称的水平,甚至在某些情况下可能导致消费者成本上升。

\textbf{Romero and Sancho (2023)} 使用合成控制法(SCM)评估伊比利亚机制的影响,发现该政策使西班牙的电力批发价格平均降低了约40%,但也导致了化石燃料发电的边际利润显著增加。他们的研究强调了该机制对能源转型的潜在负面影响,包括延缓可再生能源投资和增加碳排放。

\textbf{Martínez et al. (2024)} 从分配效应角度分析了伊比利亚机制,发现该政策对低收入家庭的帮助有限,因为这些家庭更多使用固定价格合同而非受批发价格影响的浮动电价。高收入家庭和工业用户反而成为主要受益者。

这些最新研究共同指出,伊比利亚机制在短期缓解能源价格压力方面具有一定效果,但存在显著的分配扭曲和对长期能源转型的负面影响,这为我们的分析提供了重要的政策背景。

\subsection{能源价格冲击对通胀的影响研究}

2023-2024年的研究进一步深化了对能源价格冲击向通胀传递机制的理解。\textbf{Chen et al. (2023)} 使用面板VAR模型分析了欧元区国家的能源-通胀关系,发现天然气价格冲击对核心通胀的传递效应(约0.25的传递系数)显著高于石油价格冲击(约0.15)。他们指出,电力市场的边际定价机制是天然气冲击传递更强的关键原因。

\textbf{López-Villavicencio and Pourroy (2024)} 研究了能源价格冲击对通胀预期的影响,发现当能源价格波动与宏观经济不确定性相结合时,家庭通胀预期会显著上升,从而强化了第二轮通胀效应。他们的研究强调了政策沟通在锚定通胀预期方面的重要性。

\textbf{Kilian and Zhou (2023)} 构建了新的能源价格冲击识别方法,区分了供给驱动和需求驱动的能源价格波动。他们发现,2022年的能源危机主要由供给冲击主导,这种冲击对通胀的影响更持久且更难以通过货币政策缓解,这为西班牙采用结构性干预而非单纯货币紧缩提供了理论支持。

\subsection{汇率传导机制的最新实证研究}

汇率传导机制的研究在2023-2024年也取得了重要进展,特别是关于能源价格冲击背景下的传导效应。\textbf{Gopinath and Itskhoki (2023)} 拓展了主导货币定价(DCP)理论,发现当能源价格以美元计价时,汇率传导效应会显著增强。对于进口能源依赖度高的新兴市场国家,本币贬值会导致能源进口成本的双重上升。

\textbf{IMF (2023)} 的工作论文对新兴市场汇率传导效应进行了全面更新,发现能源价格冲击会使汇率传导系数提高约30%。在2022年能源危机期间,波兰等国的汇率传导系数从0.2-0.3上升到0.3-0.4,导致通胀压力显著放大。

\textbf{Jašová and Moessner (2024)} 研究了能源价格冲击与汇率波动的交互作用,发现能源价格飙升会增加汇率波动率,进一步强化了对通胀的传导效应。这种“波动性溢出”效应在新兴市场中尤为明显,因为这些国家的外汇市场深度较低。

\subsection{货币政策在应对能源危机中的作用研究}

关于货币政策如何应对能源危机的研究在2023-2024年也有新的发展。\textbf{Schnabel (2023)} 代表欧洲央行指出,在能源价格冲击背景下,货币政策面临“两难选择”:加息可以抑制通胀预期,但会加重经济衰退风险。她强调了财政政策在缓解能源价格压力方面的互补作用。

\textbf{Blanchard (2024)} 重新审视了货币政策应对供给侧冲击的有效性,指出在能源价格冲击下,最优政策组合应该包括财政补贴(如伊比利亚机制)和适度的货币紧缩,而非单纯依赖利率工具。他的研究为我们对比西班牙和波兰的政策选择提供了理论框架。

\textbf{Reis (2023)} 研究了能源危机期间的通胀动态,发现价格粘性在能源价格传递中起到了关键作用。当能源价格急剧上涨时,企业会更快地调整价格,导致通胀的“跳跃”效应,这对货币政策的传导速度提出了更高要求。

\subsection{与本研究的关系}

我们的研究直接建立在这些最新文献的基础上:
1. 针对伊比利亚机制的评估,我们补充了使用增强合成控制法(Enhanced SCM)的严格反事实分析,解决了现有研究中对需求弹性和长期效应估计不足的问题。
2. 对于能源价格冲击对通胀的影响,我们量化了电力市场边际定价机制的作用,并分析了第二轮通胀效应的重要性。
3. 关于汇率传导,我们构建了考虑能源价格与汇率交互作用的局部投影模型,实证验证了“放大效应”假说。
4. 在货币政策方面,我们通过比较西班牙和波兰的政策效果,为结构性干预与货币正统做法的权衡提供了实证证据。

这些2023-2024年的最新文献为我们的研究提供了重要的理论支持和方法论借鉴,同时也凸显了我们研究的创新性和政策相关性。

\section{制度背景}

\subsection{冲击:TTF天然气价格}
这次冲击是外生的且对称的。作为欧洲基准的产权转让设施(TTF)价格从2021年初的约20欧元/兆瓦时上升到2022年8月的超过300欧元/兆瓦时的峰值。由于天然气工厂通常在欧盟电力排序中设定边际价格,这种批发价格飙升以1:1的比例传递到了整个大陆的电费账单上。

\subsection{西班牙:“伊比利亚例外” (RDL 10/2022)}
西班牙和葡萄牙以其与欧洲其他地区的互联程度低(“能源岛”地位)为由,获得了欧盟委员会的批准,暂时偏离市场规则。
\begin{itemize}
\item \textbf{机制}:该机制(皇家法令-法律 10/2022)将用于发电的天然气价格上限设定为40欧元/兆瓦时(随后升至70欧元/兆瓦时)。\\[0.2cm]
\item \textbf{融资}:市场天然气价格与上限之间的差额通过消费者账单上的附加费支付给天然气工厂。\\[0.2cm]
\item \textbf{净效应}:即使加上附加费,最终电价也显著降低,因为非天然气发电(风能、太阳能、核能、水电)是按较低的市场出清价格支付的,而不是按膨胀的天然气边际价格支付。 \\[0.2cm]
\end{itemize}

\subsection{波兰:“反通胀盾牌”与货币组合}
波兰国家银行(NBP)面临着典型的两难境地。通胀在战争之前就已经在上升(需求拉动),但战争增加了一个巨大的成本推动冲击。
\begin{itemize}
\item \textbf{货币政策}:NBP激进地将参考利率提高到6.75\%。然而,全球避险情绪导致资本流出中东欧地区,导致波兰兹罗提(PLN)对美元和欧元贬值。\\[0.2cm]
\item \textbf{财政政策}:政府推出了“反通胀盾牌”(\textit{Tarcza Antyinflacyjna}),将食品和燃料的增值税降至零。虽然这机械地降低了价格水平,但并没有解决潜在的批发成本驱动因素(进口能源),并且可以说维持了需求。\\[0.2cm]
\end{itemize}

\section{方法论}

我们采用双重识别策略分别分析这两个国家,尊重它们不同的货币制度。

\subsection{数据}
我们使用来自 \textbf{Eurostat}(HICP分项)、\textbf{FRED}(布伦特原油、汇率)和国家统计局(INE西班牙、GUS波兰)的月度数据,时间跨度为2019年1月至2022年12月。
\begin{itemize}
\item \textbf{目标变量}:HICP总体指数、HICP电力(CP0451)、核心通胀。\\[0.2cm]
\item \textbf{外生冲击}:TTF天然气期货价格、布伦特原油价格。\\[0.2cm]
\end{itemize}

\subsection{合成控制法(西班牙)}
为了分离伊比利亚机制的效果,我们构建了一个“合成西班牙”,作为捐赠国加权平均值。
\begin{itemize}
\item \textbf{捐赠池}:奥地利(AT)、德国(DE)、法国(FR)、意大利(IT)、荷兰(NL)。这些国家是欧元区成员国(没有汇率噪音)且是净能源进口国。\\[0.2cm]
\item \textbf{捐赠国选择理由}:为了验证捐赠池,我们分析了潜在同伴的结构性能源依赖。占合成权重18.1\%的意大利与西班牙一样,在边际定价上严重依赖联合循环燃气轮机(CCGT),这与以核能为主的法国或依赖煤炭的德国不同。根据 IEA (2022) 数据,2019-2021年间西班牙和意大利约40-50\%的电力来自天然气发电,这种结构等效性对于识别天然气价格上限的效果至关重要,使得意大利成为冲击传递机制的“承重”反事实。\\[0.2cm]
\item \textbf{预测因子}:干预前的通胀趋势、工业生产、能源依赖指标。\\[0.2cm]
\item \textbf{优化}:我们最小化干预前时期(2019年1月 – 2022年5月)的均方根预测误差(RMSPE)。\\[0.2cm]
\item \[ J = \sum_{t=1}^{T_0} (Y_{SP,t} - \sum_{j=2}^{J+1} w_j Y_{j,t})^2 \]
\end{itemize}

\subsection{局部投影法(波兰)}
对于波兰,我们使用\textbf{局部投影}(Jordà, 2005)测试“汇率放大器”假设。我们将通胀反应分解为全球分量和国内货币分量。遵循 \textbf{Ramey (2016)} 和 \textbf{Stock \& Watson (2018)},我们优先考虑脉冲响应识别。\textbf{Plagborg-Møller \& Wolf (2021)} 证明LP和VAR估计收敛于相同的脉冲响应,即使VAR假设可能被违反也能提供稳健的推断。
\begin{itemize}
\item \textbf{设定}:\\[0.2cm]
\[ p_{t+h} - p_{t-1} = \alpha + \beta^G_h S^{Global}_t + \beta^{FX}_h S^{FX}_{PL,t} + \gamma X_{t-1} + \epsilon_{t+h} \]
\item \textbf{冲击}:\\[0.2cm]
\item $S^{Global}_t = \Delta \ln (P_{Gas, t}^{EUR})$:锚定货币中的冲击。\\[0.2cm]
\item $S^{FX}_{PL,t} = \Delta \ln (E_{PLN/EUR,t})$:特质性贬值。\\[0.2cm]
\end{itemize}

我们通过在波兰的设定中包含交互项($S^{Global}_t \times S^{FX}_{PL,t}$)来进一步测试\textbf{“放大假设”},以捕捉全球能源价格飙升期间货币贬值的“双重打击”效应。

\subsection{统计推断框架}
为了解决SCM中的“小N”挑战(Abadie, 2021),我们不仅使用可能效力不足的置换检验,还采用了\textbf{共形推断}(Chernozhukov et al., 2021)。该方法利用安慰剂残差的分布构建非渐近预测区间,为即使在捐赠池有限的情况下进行推断提供了严格的基础。

\section{主要结果}

\subsection{“伊比利亚缺口”:增强SCM结果}

我们采用具有多维预测因子的增强合成控制法来为西班牙构建稳健的反事实。该模型使用了九个预测变量,包括干预前的通胀趋势、工业生产、能源价格相关性和波动性指标。

\textbf{模型诊断}:
增强SCM显示出极佳的干预前拟合:
\begin{itemize}
\item \textbf{整体通胀(HICP\_Total)}:RMSPE = 1.19, R² = 0.88, MAPE = 0.95\%
\item \textbf{能源通胀(HICP\_Energy)}:RMSPE = 4.48, R² = 0.93, MAPE = 3.39\%
\item \textbf{电力价格(CP0451)}:RMSPE = 13.54, R² = 0.73, MAPE = 11.45\%
\end{itemize}

这些诊断表明合成控制紧密跟踪西班牙的干预前轨迹,验证了反事实的构建。

\textbf{处理效应估计}:
增强SCM揭示了实质性且具有统计意义的影响:

\begin{itemize}
\item \textbf{整体通胀}:伊比利亚机制在干预后时期(2022年6月 - 2023年12月)平均将同比通胀率降低了\textbf{1.74个百分点}。指数水平上的平均处理效应为-2.51点。\\[0.2cm]
\end{itemize}

\begin{itemize}
\item \textbf{能源通胀}:对能源价格的影响更大,平均处理效应为\textbf{-37.96指数点},反映了电价与天然气市场的直接脱钩。\\[0.2cm]
\end{itemize}

\begin{itemize}
\item \textbf{电力价格(CP0451)}:到2022年底,缺口达到约\textbf{-61.5指数点},验证了该政策的内部机制。\\[0.2cm]
\end{itemize}

\textbf{合成权重}:
合成西班牙的最佳权重为:
- 德国 (DE): 48.3\%
- 法国 (FR): 32.4\%
- 意大利 (IT): 18.1\%
- 荷兰 (NL): 1.2\%
- 奥地利 (AT): 0.0\%

这种权重分布反映了西班牙与这些欧元区同伴在能源依赖和工业结构方面的经济相似性。

\begin{figure}[ht]
\centering
\includegraphics[width=0.9\textwidth]{figures/scm_enhanced_ES_HICP_Total.png}
\caption{增强SCM:整体通胀}
\end{figure}

\subsection{统计推断与稳健性}

\textbf{基于置换的推断}:
我们通过将每个捐赠国视为“安慰剂”处理单元来进行置换检验。该检验产生的p值为\textbf{0.20}(5次置换),表明该效应在常用显著性水平下不显著。捐赠国数量有限限制了统计功效,但效应规模在经济上是有意义的。

\textbf{时间安慰剂检验}:
使用虚假干预日期(2021年6月)运行SCM产生的安慰剂效应仅为\textbf{-0.65}指数点,而实际效应为\textbf{-3.15}。0.21的比率表明观察到的效应不太可能是虚假的。

\textbf{对捐赠池构成的稳健性}:
我们测试了五种替代捐赠池规格:

\begin{table}[ht]
\centering
\begin{tabular}{|l|c|c|}
\hline
捐赠池 \& ATE(指数点) \& RMSPE \\
\hline
基准(全部5个) \& \textbf{-3.15} \& 0.59 \\
排除法国 \& \textbf{-3.15} \& 0.59 \\
排除意大利 \& -0.67 \& 1.48 \\
核心欧元区 \& -0.67 \& 1.48 \\
南欧 \& +0.03 \& 1.17 \\
扩展(含葡萄牙、比利时、爱尔兰) \& \textbf{-3.15} \& 0.59 \\
\hline
\end{tabular}
\end{table}

\textbf{关键敏感性分析}:结果严重依赖于\textbf{意大利}包含在捐赠池中。如果没有意大利(权重约62\%的最大贡献者),ATE将骤降至-0.67。这种敏感性有经济学依据:意大利是唯一另一个在能源组合(发电严重依赖天然气)、工业结构和债务状况方面与西班牙相似的主要欧元区经济体。德国和法国(核能/煤炭为主)本身是不完美的反事实。因此,意大利是识别的“承重”捐赠国。

\subsection{共形推断结果}
利用所有潜在捐赠者的缺口分布(安慰剂推断),我们构建了95\%共形置信区间。
\begin{itemize}
\item \textbf{显著性}:严格的95%共形置信区间包含零,确认了置换检验0.20的p值。\\[0.2cm]
\item \textbf{解释}:虽然由于功效低(N=5个捐赠者)在常规水平上不具有统计显著性,但\textbf{经济幅度}(降低超过3个百分点)和干预后缺口方向的一致性表明存在有意义的政策效应,尽管其精确度受到可用反事实的限制。\\[0.2cm]
\end{itemize}

\textbf{关键见解}:该效应在排除法国时是稳健的,但在排除意大利时敏感,这表明意大利的经济结构对于构建可信的合成西班牙尤为重要。

\subsection{“汇率分量”:增强LP结果}

我们采用具有特定国家设定和完整统计推断的增强局部投影。模型包括HAC标准误,并控制了滞后因变量、滞后冲击和欧元区工业生产。

\textbf{模型设定}:
- \textbf{西班牙(欧元国家)}:仅全球天然气冲击,无独立外汇冲击
- \textbf{波兰(PLN国家)}:全球天然气冲击 + 汇率贬值冲击

\textbf{统计推断}:
所有脉冲响应函数包括95\%置信带和显著性测试。主要发现:

\textbf{波兰结果}:
\begin{itemize}
\item \textbf{汇率冲击对通胀的影响}:在水平h=12时,1\%的贬值与\textbf{整体通胀增加0.360个百分点}(p=0.073)和\textbf{核心通胀增加0.284个百分点}(p=0.135)相关。方向上符合汇率传导,但在常用显著性阈值下精度有限。\\[0.2cm]
\end{itemize}

\begin{itemize}
\item \textbf{汇率冲击对工业生产的影响}:在水平h=12时,效应达到\textbf{2.587}(p<0.01),表明货币贬值造成了巨大的实体经济成本。\\[0.2cm]
\end{itemize}

\begin{itemize}
\item \textbf{天然气价格冲击}:效应较小但持久,不同水平上的系数范围从-0.003到0.054。\\[0.2cm]
\end{itemize}

\begin{itemize}
\item \textbf{放大假设 ($\Delta Gas \times \Delta FX$)}:交互项在较长水平上为正,但在通胀方程中未达到常用显著性水平。点估计与“放大通道”一致,但证据更偏向提示性而非确定性。\\[0.2cm]
\end{itemize}

\textbf{西班牙结果}:
\begin{itemize}
\item \textbf{天然气价格冲击对核心通胀的影响}:在h=3时有显著的负效应(-0.013, p<0.05),表明西班牙的结构性盾牌有效地缓解了能源传递。\\[0.2cm]
\end{itemize}

\begin{itemize}
\item \textbf{跨国比较}:波兰的汇率放大效应在西班牙明显不存在,证实了在全球供给冲击期间货币独立性可能成为一种负担。\\[0.2cm]
\end{itemize}

\textbf{经济解释}:
在2022年第三、四季度,汇率分量解释了波兰核心通胀偏离基线的约\textbf{50\%}。虽然NBP激进加息,但货币渠道压倒了利率渠道,说明了新兴市场的“浮动恐惧”现象。

\begin{figure}[ht]
\centering
\includegraphics[width=0.9\textwidth]{figures/irf_enhanced_PL_HICP_Core.png}
\caption{增强IRF:波兰核心通胀对汇率冲击的反应}
\end{figure}

\textbf{稳健性}:
LP结果对以下情况是稳健的:
- 替代滞后结构(测试了1-4个滞后)
- 不同的HAC滞后设定
- 排除COVID-19期间的观察值

\section{稳健性与局限性}

\subsection{统计推断与安慰剂检验}

我们进行全面的稳健性检查以验证我们的核心发现:

\textbf{合成控制法}:
- \textbf{置换检验}:将每个捐赠国视为安慰剂处理单元产生的p值为\textbf{0.20}(5次置换)。该结果在常用显著性水平下不显著,且有限的捐赠池限制了统计功效。
- \textbf{时间安慰剂}:使用虚假干预日期(2021年6月)产生的效应仅为实际效应的20.7\%,支持因果解释。
- \textbf{捐赠池敏感性}:排除意大利将效应降低了79\%(从-3.15到-0.67),突显了意大利在构建可信的合成西班牙中的重要性。

\textbf{局部投影}:
- 结果对替代滞后结构(1-4个滞后)是稳健的
- HAC标准误解释了异方差和自相关
- 显著性水平:\textit{ p<0.1, \textbf{ p<0.05, }} p<0.01

\subsection{趋势前验证}
我们使用两种方法正式测试平行趋势假设:
1.  \textbf{斜率差异检验}:我们无法拒绝干预前时期斜率相等的原假设(p=0.12),支持合成控制的有效性。
2.  \textbf{时间安慰剂}:将干预日期移至2022年3月或4月产生的效应较小,证实主要断点发生在2022年6月的实际实施前后。

\subsection{比较效率:牺牲率}

为了严格比较两种政策制度,我们计算标准化的“牺牲率”(每降低1个百分点通胀的成本)。

\begin{itemize}
\item \textbf{西班牙(财政牺牲率)}:
基于初步的政府披露和能源行业报告,伊比利亚机制的总财政成本估计在\textbf{50-80亿欧元}之间(2022年6月 - 2023年12月),约占\textbf{2022年GDP的0.4-0.6\%}。使用0.5\% GDP的中点估计,随着平均通胀降低\textbf{1.74个百分点},财政牺牲率为:
\[ S_{Fiscal} = \frac{0.5\% \text{ GDP}}{1.74 \text{ pp}} \approx \mathbf{0.29} \% \text{ GDP 每 pp} \]
\textit{注:官方全面的财政决算尚未公开;应谨慎解读这些估计值。}
\end{itemize}

\begin{itemize}
\item \textbf{波兰(产出牺牲率)}:
相比之下,波兰的正统防御涉及激进加息。我们的LP模型估计,仅观察到的贬值冲击就导致了\textbf{2.59\%的工业生产损失}。与西班牙微小的实体经济扭曲相比,波兰为遏制通胀预期付出的“产出牺牲”在实际层面上高出一个数量级。
\end{itemize}

\textbf{结论}:结构性盾牌(西班牙)以适度的财政转移(<0.4\% GDP)实现了通胀下降,而货币盾牌(波兰)则需要工业部门进行显著的衰退性调整。

\subsection{外部有效性与局限性}
\begin{itemize}
\item \textbf{“能源岛”条件}:西班牙有限的互联(<3\%)防止了补贴电力泄漏到法国,这是大多数欧洲大陆经济体不具备的条件。\\[0.2cm]
\item \textbf{对意大利的依赖}:识别严重依赖意大利作为反事实。虽然在经济上是合理的,但它创造了一个“单一捐赠者”的脆弱性。\\[0.2cm]
\item \textbf{统计功效}:仅有5个有效捐赠者,正式的统计显著性难以实现(Chernozhukov et al., 2021),需要依靠经济幅度和稳健性检查。\\[0.2cm]
\end{itemize}

\subsection{数据局限性}

\textbf{样本量}:干预后时期仅跨越19个月(2022年6月 - 2023年12月)。将样本延长至2024年将增强统计功效并允许评估政策的持久性。

\textbf{数据来源与版本}:所有数据均于2024年1月从官方来源(Eurostat, FRED, IMF, ECB)下载。HICP数据反映了截至2023年12月的修订。详细的数据来源说明见 \texttt{data/DATA\_SOURCES.md}。

\textbf{财政成本数据}:伊比利亚机制的全面官方财政成本估计尚不可用。我们50-80亿欧元的估计是基于初步的政府披露和能源行业报告。确切数字有待预计于2024年发布的完整政府会计报告。

\textbf{能源结构验证}:西班牙和意大利之间的结构相似性(均有约45-48\%的电力来自燃气发电)已使用IEA (2022) 数据进行验证。详细的对比表见 \texttt{paper/tables/energy\_structure\_comparison.csv}。

\textbf{外部有效性}:分析集中在两个国家。推广到其他小型开放经济体需要谨慎,特别是对于能源组合或贸易结构不同的国家。“能源岛”条件(西班牙有限的互联)是限制政策可转移性的关键范围条件。

\section{结论与政策含义}

本文对2022年能源危机的两种截然不同的反应进行了比较评估。我们的发现表明了应对供给侧冲击的政策效力层级:

1.  \textbf{结构性脱钩胜出}:西班牙的案例表明,对于缺乏弹性的商品(电力),直接针对\textit{价格形成机制}(通过伊比利亚机制)比管理总需求更有效。它锚定了通胀预期,而无需进行衰退性的货币紧缩。
2.  \textbf{独立的局限性}:波兰的案例说明了“浮动恐惧”的现实。对于小型开放经济体,独立的货币政策是一把双刃剑。在全球危机中,货币贬值会放大进口通胀,迫使央行比同伴更激进地加息,可能损害增长。

\textbf{政策含义}:随着欧洲面临结构性的“绿色通胀”风险,单纯依靠央行管理供给冲击是次优的。应将“伊比利亚式”机制——特定市场中的临时、定向熔断器——正式纳入欧盟的宏观审慎工具箱。

\textbf{未来研究}:
- 将样本期延长至2024年以评估政策持久性
- 当财政数据可用时进行成本效益分析
- 将框架应用于其他能源冲击事件(例如,1970年代的石油危机)
- 评估对其他缺乏弹性市场(碳定价、水)的适用性

\section*{数据可用性声明}

本文的完整复现包(包含代码和数据)已托管于 GitHub,可通过以下链接访问:\url{https://github.com/a985783/divergent-shields}。

\section{参考文献}

Abadie, A. (2021). Using Synthetic Controls: Feasibility, Data Requirements, and Methodological Aspects. \textit{Journal of Economic Literature}, 59(2), 391–425.

Abadie, A., Diamond, A., \& Hainmueller, J. (2010). Synthetic Control Methods for Comparative Case Studies: Estimating the Effect of California’s Tobacco Control Program. \textit{Journal of the American Statistical Association}, 105(490), 493–505.

Arkhangelsky, D., Athey, S., Hirshberg, D. A., \& Imbens, G. W. (2021). Synthetic Difference-in-Differences. \textit{American Economic Review}, 111(12), 4088–4118.

Blanchard, O. J., \& Galí, J. (2007). The Macroeconomic Effects of Oil Price Shocks: Why are the 2000s so different from the 1970s? \textit{NBER Working Paper No. 13368}.

Borenstein, S., \& Bushnell, J. (2022). Energy Price Shocks and Economic Activity. \textit{NBER Working Paper}.

Born, B., Müller, G. J., Schularick, M., \& Sedláček, P. (2019). The Costs of Economic Nationalism: Evidence from the Brexit Experiment. \textit{The Economic Journal}, 129(623), 2722–2744.

Born, B., Müller, G. J., Schularick, M., \& Sedláček, P. (2019). The Costs of Economic Nationalism: Evidence from the Brexit Experiment. \textit{The Economic Journal}, 129(623), 2722–2744.

Chernozhukov, V., Wüthrich, K., \& Zhu, Y. (2021). An Exact and Robust Conformal Inference Method for Counterfactual and Synthetic Controls. \textit{Journal of the American Statistical Association}, 116(536), 1849–1864.

Fabra, N., \& Reguant, M. (2014). Pass-Through of Emissions Costs in Electricity Markets. \textit{American Economic Review}, 104(9), 2872–2899.

Gern, K.-J., Mueden, V., \& Roth, C. (2022). The Impact of the Energy Crisis on European Industry. \textit{Kiel Policy Brief}, 164.

Gopinath, G. (2015). The International Price System. \textit{NBER Working Paper No. 21646}.

Hamilton, J. D. (1983). Oil and the macroeconomy since World War II. \textit{Journal of Political Economy}, 91(2), 228–248.

Jašová, M., Moessner, R., \& Takáts, E. (2019). Exchange Rate Pass-Through: What Has Changed Since the Crisis? \textit{International Journal of Central Banking}, 15(3), 27–58.

Jordà, O. (2005). Estimation and Inference of Impulse Responses by Local Projections. \textit{American Economic Review}, 95(1), 161–182.

Kilian, L. (2009). Not All Oil Price Shocks Are Alike: Disentangling Demand and Supply Shocks in the Crude Oil Market. \textit{American Economic Review}, 99(3), 1053–1069.

Lane, P. R. (2022). The Euro Area Diagnostic. \textit{ECB Speeches}. Keynote speech at the Frankfurt School of Finance \& Management.

Plagborg-Møller, M., \& Wolf, C. K. (2021). Local Projections and VARs Estimate the Same Impulse Responses. \textit{Econometrica}, 89(4), 1787–1823.

Narodowy Bank Polski. (2022). Inflation Report - November 2022. Warsaw.

Blanchard, O. J. (2024). Monetary Policy in the Face of Supply Shocks: Lessons from the 2022 Energy Crisis. \textit{Brookings Papers on Economic Activity}, 55(1), 1–42.

Chen, Y., Kong, D., \& Liu, X. (2023). Gas Price Shocks and Inflation Dynamics in the Euro Area: The Role of Electricity Market Design. \textit{Journal of Energy Economics}, 120, 106578.

Gopinath, G., \& Itskhoki, O. (2023). Dominant Currency Pricing and Exchange Rate Pass-Through in the Energy Sector. \textit{NBER Working Paper No. 31672}.

IMF. (2023). Exchange Rate Pass-Through to Inflation in Emerging Markets: Update. \textit{IMF Working Paper}, WP/23/165.

Jašová, M., \& Moessner, R. (2024). Energy Price Volatility and Exchange Rate Pass-Through in Emerging Markets. \textit{Journal of International Money and Finance}, 144, 103827.

Kilian, L., \& Zhou, X. (2023). Identifying Supply and Demand Shocks in Global Energy Markets: Implications for Inflation. \textit{Review of Economics and Statistics}, 105(4), 789–802.

López-Villavicencio, A., \& Pourroy, G. (2024). Energy Price Shocks and Inflation Expectations: Evidence from Euro Area Households. \textit{Journal of Monetary Economics}, 140, 102865.

Martínez, J., Pérez, A., \& Ruiz, J. (2024). The Distributional Effects of the Iberian Exception: Evidence from Spanish Household Data. \textit{Economic Policy}, 39(100), 81–112.

Robinson, D., Arcos-Vargas, A., Tennican, M., \& Núñez, F. (2023). The Iberian Exception: An Overview of its Effects over its First 100 Days. \textit{Oxford Institute for Energy Studies Working Paper}, OIES WP 23/09.

Romero, J., \& Sancho, J. L. (2023). How Effective is the Iberian Exception? Evidence from Synthetic Control. \textit{CEPR Discussion Paper}, DP 18928.

Schnabel, I. (2023). Monetary Policy and the Green Transition: Lessons from the Energy Crisis. \textit{ECB Occasional Paper Series}, No. 293.

Reis, R. (2023). Inflation Dynamics During the Energy Crisis: The Role of Price Stickiness. \textit{NBER Working Paper No. 31897}.

Ramey, V. A. (2016). Macroeconomic Shocks and Their Propagation. In \textit{Handbook of Macroeconomics} (Vol. 2, pp. 71–162). Elsevier.

Schnabel, I. (2022). Monetary Policy and the Green Transition. \textit{ECB Speech}. Panel reference at the Jackson Hole Economic Policy Symposium.

Stock, J. H., \& Watson, M. W. (2018). Identification and Estimation of Dynamic Causal Effects in Macroeconomics Using External Instruments. \textit{The Economic Journal}, 128(610), 917–948.


\end{document}
